\documentclass{article}
\usepackage[a4paper, margin=0.2in]{geometry}
\usepackage{amsmath,amssymb,amsfonts,amsthm}
\usepackage[backend=biber]{biblatex}
\usepackage{hyperref}
\usepackage{caption}

\usepackage{verbatim}
%copied from https://tex.stackexchange.com/questions/309042/how-can-i-redefine-verbatim-to-wrap-lines-add-break-before-or-after-spaces
\makeatletter
\newenvironment{myverb}
 {\def\@xobeysp{\ }\verbatim\rightskip=0pt plus 6em\relax}
 {\endverbatim}
\makeatother

\usepackage[utf8]{inputenc}
\usepackage[T1]{fontenc}


\author{Haiyang He}
\title{Milestone 3}

\begin{document}
\maketitle
\paragraph{Required information}
    \begin{itemize}
        \item name: Haiyang He
        \item UCID: 30067349
        \item email: haiyang.he@ucalgary.ca
        \item gitlab repo \url{https://gitlab.cpsc.ucalgary.ca/haiyang.he/cpsc411}
    \end{itemize}

    The information below is in \texttt{README.md} inside the repo, but we will briefly touch on how to build the project, and where are the test cases.

    You can build the code generation by \texttt{make codegen}, and this should produce a binary file named \texttt{./codegen.out}. Then we can run it by giving it a file name \texttt{./codegen.out <filename>}.
All of the test cases from Dr. Aycock's reference compiler folder is located at \texttt{./testFiles/finalReferenceTests}. 


\end{document}
